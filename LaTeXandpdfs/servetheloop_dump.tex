% file: servetheloop_dump_grande.tex
% rLoop notes "dump" 
% 
% github        : ernestyalumni
% linkedin      : ernestyalumni 
%
% This code is open-source, governed by the Creative Common license.  Use of this code is governed by the Caltech Honor Code: ``No member of the Caltech community shall take unfair advantage of any other member of the Caltech community.'' 
% 
\documentclass[10pt]{amsart}
\pdfoutput=1
\usepackage{mathtools,amssymb,lipsum,caption}

\usepackage{graphicx}
\usepackage{hyperref}
\usepackage[utf8]{inputenc}
\usepackage{listings}
\usepackage[table]{xcolor}
\usepackage{pdfpages}
%\usepackage[version=3]{mhchem}
\usepackage{mhchem}

\usepackage{tikz}
\usetikzlibrary{matrix,arrows}

\usepackage{multicol}

\hypersetup{colorlinks=true,citecolor=[rgb]{0,0.4,0}}

\oddsidemargin=15pt
\evensidemargin=5pt
\hoffset-45pt
\voffset-55pt
\topmargin=-4pt
\headsep=5pt
\textwidth=1120pt
\textheight=595pt
\paperwidth=1200pt
\paperheight=700pt
\footskip=40pt


\newtheorem{theorem}{Theorem}
\newtheorem{corollary}{Corollary}
%\newtheorem*{main}{Main Theorem}
\newtheorem{lemma}{Lemma}
\newtheorem{proposition}{Proposition}

\newtheorem{definition}{Definition}
\newtheorem{remark}{Remark}

\newenvironment{claim}[1]{\par\noindent\underline{Claim:}\space#1}{}
\newenvironment{claimproof}[1]{\par\noindent\underline{Proof:}\space#1}{\hfill $\blacksquare$}

%This defines a new command \questionhead which takes one argument and
%prints out Question #. with some space.
\newcommand{\questionhead}[1]
  {\bigskip\bigskip
   \noindent{\small\bf Question #1.}
   \bigskip}

\newcommand{\problemhead}[1]
  {
   \noindent{\small\bf Problem #1.}
   }

\newcommand{\exercisehead}[1]
  { \smallskip
   \noindent{\small\bf Exercise #1.}
  }

\newcommand{\solutionhead}[1]
  {
   \noindent{\small\bf Solution #1.}
   }


  \title{%servetheloop dump 
  \large servetheloop dump (for the rLoop) }

\author{Ernest Yeung \href{mailto:ernestyalumni@gmail.com}{ernestyalumni@gmail.com}}
\date{27 mai 2017}
\keywords{Eddy current brakes}
\begin{document}

\definecolor{darkgreen}{rgb}{0,0.4,0}
\lstset{language=Python,
 frame=bottomline,
 basicstyle=\scriptsize,
 identifierstyle=\color{blue},
 keywordstyle=\bfseries,
 commentstyle=\color{darkgreen},
 stringstyle=\color{red},
 }
%\lstlistoflistings

\maketitle


\noindent gmail        : ernestyalumni \\
linkedin     : ernestyalumni \\
twitter      : ernestyalumni \\

\setcounter{tocdepth}{1}
\tableofcontents


\begin{multicols*}{2}

  



\begin{abstract}
servetheloop notes "dump" - I dump all my notes, including things that I tried but are wrong, here.  

\end{abstract}

\part{Eddy Currents, Eddy Current Braking}

\section{Eddy Currents}

\emph{Keywords:} Eddy currents;

cf. Smythe (1968), Ch. X (his Ch. 10) \cite{Smyt1968}

Assume Maxwell's "displacement current" is negligible; this is ok if frequencies are such that wavelength $\lambda$ large compared to dimensions of apparatus $L$.  $\lambda \gg L$ or $\frac{c}{\nu} \gg L$.  

I will write down the "vector calculus" formulation of electrodynamics, along side Maxwell's equations, or electrodynamics, over spacetime manifold $M$.  The latter formulation should specialize to the "vector calculus" formulation.  

From 
\[
\text{curl} \mathbf{E} = -\frac{d\mathbf{B}}{dt} \, (SI) \qquad \, \text{curl}\mathbf{E} = -\frac{1}{c} \frac{ \partial \mathbf{B}}{ \partial t} \, (cgs) \text{ or } \frac{ \partial B}{ \partial t} + \mathbf{d}E = 0 
\]
Suppose $B = \text{curl} A$ or $B=\mathbf{d}A$ (EY 20170528: is this where the assumption above about $\lambda \gg L$ comes in?), then
\[
\begin{gathered}
	-\frac{\partial B}{ \partial t} = \mathbf{d}E \xrightarrow{ \int_S } \int_S \mathbf{d}E = \int_{ \partial S} E = -\int_S \frac{ \partial B}{ \partial t} \xrightarrow{ B = \mathbf{d}A } -\int_S \frac{ \partial }{ \partial t} \mathbf{d}A \xrightarrow{ \text{ flat space } } \int_S \mathbf{d}E = -\int_S \mathbf{d} \frac{ \partial A}{ \partial t}	
\end{gathered}
\]
and so
\begin{equation}
	\mathbf{E} = \frac{-\partial \mathbf{A}}{ \partial t}
\end{equation}
up to gauge transformation, if $B = \mathbf{d}\mathbf{A} = \text{curl}\mathbf{A}$

Since this $\mathbf{E}$ field is formed in a conductor, Ohm's law applies.  Let's review Ohm's law.  Smythe (1968) refers to its 6.02 Ohm's Law - Resistivity section \cite{Smyt1968}.  Indeed, in a lab, the definition of resistance can be defined as this ratio:
\begin{equation}
R_{AB} := \frac{ -\int_A^B \mathbf{E} }{ I_{AB} } = \frac{ V_A- V_B}{ I_{AB} } = \frac{ \varepsilon_{AB} }{ I_{AB} }
\end{equation}
Moving along, the right way to think about resistivity $\rho$ is to consider conductivity.  

Assume current density is linear to $\mathbf{E}$ field (as $\mathbf{E}$ field pushes charges along).  This linear response is reasonable.  

Also, assume current density $\mathbf{J}$ is uniform over $dA$ (i.e. surface $S$).  

Define 
\begin{equation}
\sigma \equiv \text{ \textbf{ conductivity } }
\end{equation}

Then the empirical relation/equation that underpins \emph{Ohm's law} is 
\begin{equation}
\mathbf{J} = \sigma \mathbf{E}
\end{equation}
and define \emph{resistivity} from there:
\begin{equation}
\sigma := \frac{1}{\rho}
\end{equation}
where $\rho $ is the \emph{resistivity}. 

Thus
\[
\begin{gathered}
	\sigma \int_A^B -\mathbf{E} = \sigma V_{AB} = \int_A^B \mathbf{J} = I_{AB} \frac{l}{A} \\
\frac{1}{\rho} R = \frac{l}{A} \text{ or } \boxed{ R = \rho \frac{l}{A} }
\end{gathered}
\]




\end{multicols*}

\begin{thebibliography}{9}


\bibitem{Hugh2000}
Scott B. Hughes.  \textbf{Magnetic braking: Finding the effective length over which the eddy currents form}.  
\href{https://drive.google.com/file/d/0Bwo3W0v5P04LX29XT2NFeVY0a1E/view}{Magnetic braking: Finding the effective length over which the eddy currents form}  


\bibitem{Smyt1968}
William R. Smythe, \textbf{Static and Dynamic Electricity}.  3rd ed. (McGraw-Hill, New York, 1968).  


\end{thebibliography}

\end{document}
