% file: servetheloop_dump_grande.tex
% rLoop notes "dump" 
% 
% github        : ernestyalumni
% linkedin      : ernestyalumni 
%
% This code is open-source, governed by the Creative Common license.  Use of this code is governed by the Caltech Honor Code: ``No member of the Caltech community shall take unfair advantage of any other member of the Caltech community.'' 
% 
\documentclass[10pt]{amsart}
\pdfoutput=1
\usepackage{mathtools,amssymb,lipsum,caption}

\usepackage{graphicx}
\usepackage{hyperref}
\usepackage[utf8]{inputenc}
\usepackage{listings}
\usepackage[table]{xcolor}
\usepackage{pdfpages}
%\usepackage[version=3]{mhchem}
\usepackage{mhchem}

\usepackage{tikz}
\usetikzlibrary{matrix,arrows}

\usepackage{multicol}

\hypersetup{colorlinks=true,citecolor=[rgb]{0,0.4,0}}

\oddsidemargin=15pt
\evensidemargin=5pt
\hoffset-45pt
\voffset-55pt
\topmargin=-4pt
\headsep=5pt
\textwidth=1120pt
\textheight=595pt
\paperwidth=1200pt
\paperheight=700pt
\footskip=40pt


\newtheorem{theorem}{Theorem}
\newtheorem{corollary}{Corollary}
%\newtheorem*{main}{Main Theorem}
\newtheorem{lemma}{Lemma}
\newtheorem{proposition}{Proposition}

\newtheorem{definition}{Definition}
\newtheorem{remark}{Remark}

\newenvironment{claim}[1]{\par\noindent\underline{Claim:}\space#1}{}
\newenvironment{claimproof}[1]{\par\noindent\underline{Proof:}\space#1}{\hfill $\blacksquare$}

%This defines a new command \questionhead which takes one argument and
%prints out Question #. with some space.
\newcommand{\questionhead}[1]
  {\bigskip\bigskip
   \noindent{\small\bf Question #1.}
   \bigskip}

\newcommand{\problemhead}[1]
  {
   \noindent{\small\bf Problem #1.}
   }

\newcommand{\exercisehead}[1]
  { \smallskip
   \noindent{\small\bf Exercise #1.}
  }

\newcommand{\solutionhead}[1]
  {
   \noindent{\small\bf Solution #1.}
   }


  \title{%servetheloop dump 
  \large $\#$servetheloop dump (for the rLoop) }

\author{Ernest Yeung \href{mailto:ernestyalumni@gmail.com}{ernestyalumni@gmail.com}}
\date{27 mai 2017}
\keywords{Eddy current brakes}
\begin{document}

\definecolor{darkgreen}{rgb}{0,0.4,0}
\lstset{language=Python,
 frame=bottomline,
 basicstyle=\scriptsize,
 identifierstyle=\color{blue},
 keywordstyle=\bfseries,
 commentstyle=\color{darkgreen},
 stringstyle=\color{red},
 }
%\lstlistoflistings

\maketitle


\noindent gmail        : ernestyalumni \\
linkedin     : ernestyalumni \\
twitter      : ernestyalumni \\

\setcounter{tocdepth}{1}
\tableofcontents


\begin{multicols*}{2}

  



\begin{abstract}
servetheloop notes "dump" - I dump all my notes, including things that I tried but are wrong, here.  

\end{abstract}

\part{Eddy Currents, Eddy Current Braking}

\section{Eddy Currents}

\emph{Keywords:} Eddy currents;

cf. Smythe (1968), Ch. X (his Ch. 10) \cite{Smyt1968}

Assume Maxwell's "displacement current" is negligible; this is ok if frequencies are such that wavelength $\lambda$ large compared to dimensions of apparatus $L$.  $\lambda \gg L$ or $\frac{c}{\nu} \gg L$.  

I will write down the "vector calculus" formulation of electrodynamics, along side Maxwell's equations, or electrodynamics, over spacetime manifold $M$.  The latter formulation should specialize to the "vector calculus" formulation.  

From 
\[
\text{curl} \mathbf{E} = -\frac{d\mathbf{B}}{dt} \, (SI) \qquad \, \text{curl}\mathbf{E} = -\frac{1}{c} \frac{ \partial \mathbf{B}}{ \partial t} \, (cgs) \text{ or } \frac{ \partial B}{ \partial t} + \mathbf{d}E = 0 
\]
Suppose $B = \text{curl} A$ or $B=\mathbf{d}A$ (EY 20170528: is this where the assumption above about $\lambda \gg L$ comes in?), then
\[
\begin{gathered}
	-\frac{\partial B}{ \partial t} = \mathbf{d}E \xrightarrow{ \int_S } \int_S \mathbf{d}E = \int_{ \partial S} E = -\int_S \frac{ \partial B}{ \partial t} \xrightarrow{ B = \mathbf{d}A } -\int_S \frac{ \partial }{ \partial t} \mathbf{d}A \xrightarrow{ \text{ flat space } } \int_S \mathbf{d}E = -\int_S \mathbf{d} \frac{ \partial A}{ \partial t}	
\end{gathered}
\]
and so
\begin{equation}
	\mathbf{E} = \frac{-\partial \mathbf{A}}{ \partial t}
\end{equation}
up to gauge transformation, if $B = \mathbf{d}\mathbf{A} = \text{curl}\mathbf{A}$

Since this $\mathbf{E}$ field is formed in a conductor, Ohm's law applies.  Let's review Ohm's law.  Smythe (1968) refers to its 6.02 Ohm's Law - Resistivity section \cite{Smyt1968}.  Indeed, in a lab, the definition of resistance can be defined as this ratio:
\begin{equation}
R_{AB} := \frac{ -\int_A^B \mathbf{E} }{ I_{AB} } = \frac{ V_A- V_B}{ I_{AB} } = \frac{ \varepsilon_{AB} }{ I_{AB} }
\end{equation}
Moving along, the right way to think about resistivity $\rho$ is to consider conductivity.  

Assume current density is linear to $\mathbf{E}$ field (as $\mathbf{E}$ field pushes charges along).  This linear response is reasonable.  

Also, assume current density $\mathbf{J}$ is uniform over infinitesimal surface area $dA$ (i.e. surface $S$).  

Define 
\begin{equation}
\sigma \equiv \text{ \textbf{ conductivity } }
\end{equation}

Then the empirical relation/equation that underpins \emph{Ohm's law} is 
\begin{equation}
\mathbf{J} = \sigma \mathbf{E}
\end{equation}
and define \emph{resistivity} from there:
\begin{equation}
\sigma := \frac{1}{\rho}
\end{equation}
where $\rho $ is the \emph{resistivity}. 

Thus
\[
\begin{gathered}
	\sigma \int_A^B -\mathbf{E} = \sigma V_{AB} = \int_A^B \mathbf{J} = I_{AB} \frac{l}{A} \\
\frac{1}{\rho} R = \frac{l}{A} \text{ or } \boxed{ R = \rho \frac{l}{A} }
\end{gathered}
\]


From Maxwell's Equations, 
\begin{equation}
	\mathbf{\delta}(B - 4\pi \mathbf{M} ) = \frac{4\pi }{c} J_{\text{free}} 
\end{equation}	

If $B=H+4\pi M = (1+4\pi \chi_m) H = \mu H$, 
then
\begin{equation}
\mathbf{\delta}H = \mathbf{\delta} \frac{B}{\mu} = \frac{ 4\pi }{c} J_{\text{free} }\text{ or } \mathbf{\delta} B = \mu \frac{4\pi }{c} J_{\text{free}} \Longleftrightarrow \text{curl} B = \mu J_{\text{free}} \qquad \, (SI)
\end{equation}
and if $B=\mathbf{d}A$ and $\mathbf{d} \mathbf{\delta}A =0$.  

%\subsubsection{Eddy Currents}

I build upon the physical setup proposed by Jackson (1998) \cite{Jack1998} in Section 5.18 "Quasi-Static Magnetic Fields in Conductors; Eddy Currents; Magnetic Diffusion."   

For a system (with characteristic) length $L$, $L$ being small, \\
compared to electromagnetic wavelength associated with dominant time scale of problem $T$, 
\[
\begin{gathered}
	f := \frac{1}{T} ; \quad \, \omega = 2\pi f ; \quad \, \omega \lambda = c \Longrightarrow \lambda = \frac{c}{ \omega } = \frac{c}{ 2\pi f } = \frac{Tc}{2\pi }  \\
\frac{L}{\lambda} = \frac{LTc}{2\pi } \gg 1
\end{gathered}
\]
From Maxwell's equations, in particular, Faraday's Law, and in its integral form (over 2-dim. \emph{closed} surface $S$), 
\begin{equation}
\begin{gathered}
\mathbf{d}E + \frac{ \partial }{ \partial t } B = 0 \text{ or } -\mathbf{d}E =\frac{ \partial B}{ \partial t} \xrightarrow{ \int_S } \int_S \frac{ \partial B}{ \partial t} = -\int_S \mathbf{d}E = -\int_{\partial S} E
\end{gathered}
\end{equation}
So on $S$, changing magnetic flux $\int \frac{ \partial B}{ \partial t}$ results in $E$ field, circulating around boundary of $S$, $\partial S$.  

We know that in a conductor, free conducting electrons get pushed around by $E$ fields, result in a current density $J$.  

$J$ is related to $E$, \emph{empirically} (by Ohm's Law)
\[
J = \sigma E
\]
where $\sigma$ is the resistivity.  

Then use the force law on this induced current $J$ from the $B$ field set up:
\[
F_{\text{net}} = \frac{1}{c} \int_S J\times B dA
\]
By working through the right-hand rule, $F_{\text{net}}$ the force on those currents induced in the conductor due to the $B$ that's there, is in the direction to help oppose changing (increasing or decreasing $\frac{\partial B}{ \partial t}$).  

To find $B$, suppose $B=dA$, i.e. $B\in H^2_{\text{deRham}}(M)$, i.e. $B=\text{curl}A$.  

For sure, 
\[
\mathbf{\delta} (B-4\pi c \mathbf{M}) = 4\pi J \Longleftrightarrow \text{curl}(B-4\pi c \mathbf{M} ) = \text{curl} H = 4\pi J
\]
Be warned now that the relation $B=\mu H$ may not be valid on all domains of interest; $\mu$ could even be a tensor! (e.g. $B_{ij} = \mu^{kl}_{ij} H_{kl}$).  However, both Jackson (1998) \cite{Jack1998} in Sec. 5.18 Quasi-Static Magnetic Fields in Conductors; Eddy Currents; Magnetic Diffusion, pp. 219, and Smythe (1968), Ch. X (his Ch. 10), pp. 368 \cite{Smyt1968}, continues on \emph{as if} this relation is linear: $B=\mu H$.  

Nevertheless, as we want to find $B$ by finding its "vector potential" $A$, we obtain a diffusion equation: 
\begin{equation}\label{Eq:EddyCurrentsAdiffusion}
\begin{gathered}
	- \mathbf{\delta}B = \mathbf{*d*d}A = (-1) \mathbf{\delta d} A = (-1)( \Delta - \mathbf{d\delta} ) A \xrightarrow{ \mathbf{d\delta} A = 0 } - \Delta A = \\
	= 4\pi \mu J = 4\pi \mu \sigma E = 4\pi \mu \sigma \left( -\frac{ \partial A}{ \partial t} \right) \\
\Longrightarrow \boxed{ \Delta A = 4\pi \mu \sigma \frac{ \partial A}{ \partial t } }
\end{gathered}
\end{equation}
where in the first 2 steps (equalities), $- \mathbf{\delta}B = \mathbf{*d*d}A = (-1) \mathbf{\delta d} A$ it's interesting to note that the codifferential $\mathbf{\delta}$ for the 2 form $B$ had to be written out explicitly, and then the codifferential for the 1-form $A$ is \emph{different} from the $\mathbf{\delta}$ for $B$ by a(n important) factor of $(-1)$; where $\mathbf{d\delta} A=0$ must be satisfied by the form $A$ takes; and where, since $B=\mathbf{d}A$,
\begin{equation}
\begin{gathered}
	\mathbf{d}E + \frac{ \partial B}{ \partial t} = \mathbf{d} E + \frac{ \partial }{ \partial t} \mathbf{d} A = \mathbf{d} \left( E+ \frac{ \partial A}{ \partial t} \right) = 0 \Longrightarrow E = -\frac{ \partial A}{ \partial t} + \text{grad}\Phi \xrightarrow{ \Phi = \text{ constant } } E = -\frac{ \partial A}{ \partial t}
\end{gathered}
\end{equation}
whereas a choice of gauge for $E$ was chosen so that $\Phi=\text{constant}$ (and so a particular form for $E$ was chosen, amongst those in the \emph{same} equivalence class of $H^1_{\text{deRham}}(M)$.  

To ensure that the differential geometry formulation is in agreement with the practical vector calculus formulation, compare Eq. \ref{Eq:EddyCurrentsAdiffusion} with Eq. (5.160) of Jackson (1998) \cite{Jack1998} and Eq. (10) in Sec. 10.00 of Smythe (1968) \cite{Smyt1968}.  

To summarize what's going on, I think one should at least understand in one's head how Maxwell's Equations apply, (and I will try to write in SI here)
\begin{equation}
\boxed{
\begin{gathered}
\int_S \frac{ \partial \mathbf{B}}{ \partial t} dA = -\oint \mathbf{E}\cdot d\mathbf{s} \Longrightarrow \mathbf{J}=\sigma \mathbf{E} \Longrightarrow \mathbf{F}_{\text{net}} = \int_S \mathbf{J} \times \mathbf{B} dA  \\
\text{ find } \mathbf{B} = ? \qquad \, \text{ using form } \mathbf{B} = \nabla \times \mathbf{A}, \\
\nabla^2 \mathbf{A} = \mu \sigma \frac{ \partial \mathbf{A} }{ \partial t} \qquad \, (SI)
\end{gathered}
}
\end{equation}
where, a change in magnetic flux over a surface $S$ over the conductor, $\int_S \frac{ \partial \mathbf{B}}{ \partial t}dA$ induces a circulation of $E$ field around $S$, $-\oint \mathbf{E} \cdot d\mathbf{s}$, and this $E$ field is pushing around \emph{free conducting charges} according to Ohm's law, $\mathbf{J} = \sigma \mathbf{E}$, with $\sigma$ being the conductivity of the conducting material, and this current density $\mathbf{J}$ is then acted upon by the prevailing $B$ field, according to the usual force law.  To find $\mathbf{B}$, one can find $\mathbf{A}$ and \emph{try} to find $\mathbf{A}$ analytically.  

Keep in mind that for $\nabla^2 \mathbf{A} = \mu \sigma \frac{ \partial \mathbf{A}}{ \partial t}$, we had used, critically, the Maxwell equation $\mathbf{\nabla} \times \mathbf{H} = \mathbf{J}$, with $\mathbf{J}$ being the \emph{induced current of free conducting charges on the conductor}.  This $\mathbf{H}$ will contribute (through linear superposition) to the $\mathbf{B}$ that could already be there due to the permanent magnet.  

What can we measure quantitatively?
\begin{itemize}
\item Can we measure $\mathbf{J}$ inside (on) the conductor?
\item Can we separate magnetization $\mathbf{M}$ of material from $\mathbf{B}$, to obtain the actual $\mathbf{B}$ (and then use linear superposition, $\mathbf{B}_{\text{total}} = \mathbf{B}_{\text{permanent magnet}} + \mathbf{B}_{\text{induced currents}}$? 
\end{itemize}

Also, keep in mind the context that the conductor at hand is the long, almost semi-infinite rectangle of a conductor, aluminum sub-rail, specified by the SpaceX Hyperloop.  Force on that will cause an equal and opposite force on the pod, with its permanent magnets attached, and thus braking the pod.   




 


\end{multicols*}

\begin{thebibliography}{9}


\bibitem{Hugh2000}
Scott B. Hughes.  \textbf{Magnetic braking: Finding the effective length over which the eddy currents form}.  
\href{https://drive.google.com/file/d/0Bwo3W0v5P04LX29XT2NFeVY0a1E/view}{Magnetic braking: Finding the effective length over which the eddy currents form}  


\bibitem{Smyt1968}
William R. Smythe, \textbf{Static and Dynamic Electricity}.  3rd ed. (McGraw-Hill, New York, 1968).  

\bibitem{Jack1998}
J.D. Jackson.  \textbf{Classical Electrodynamics} Third Edition.  Wiley.  1998.   ISBN-13: 978-0471309321


\end{thebibliography}

\end{document}
